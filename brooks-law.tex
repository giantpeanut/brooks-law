\documentclass{article}

% margins
\usepackage[top=1in, bottom=1.5in, left=1.5in, right=1.5in]{geometry}

% smart quotes
\usepackage [english]{babel}
\usepackage [autostyle, english = american]{csquotes}
\MakeOuterQuote{"}

% line spacing
\usepackage{setspace}
\singlespacing

% tiny bullet points
\renewcommand{\labelitemi}{$\vcenter{\hbox{\tiny$\bullet$}}$}

% new list environment for lists with no labels
% documentation on how this works at:
%     run `texdoc source2e`
%     look at section about lists on page 218 
\newenvironment{atomize}
    {\begin{list} {} {
            \setlength\itemindent{0pt}
            \setlength\leftmargin{10pt}
            \setlength\labelwidth{0pt}
    }}
    {\end{list}}

% info for \maketitle
\author{Dylan Holmes, Kuangyou Yao, Jonathan K}
\title{Brooks' law}
\date{February 17$^{th}$, 2014}

% todo checklist
% - - — - - - - - - - - - - — - - - - - - - - - - — - - - - - - - - - - 
% [ ] write more
% - - — - - - - - - - - - - — - - - - - - - - - - — - - - - - - - - - - 

\begin{document}

\maketitle

\section*{Abstract}
According to Brooks' Law, "adding manpower to a late software project makes it
later". Brooks' law originated from Fred Brooks' 1975 book, "The Mythical
Man-Month", and has since become commonly accepted through out the software
development community. 

Suppose in a software firm, a project manager is facing a deadline in one month.
Given that the project is only 60\% done, the project manager faces the problem
of deciding if he should hire additional engineers to pace up the process.

Brook’s law argues that by adding new engineers, experienced and efficient
engineers will have to give up part of their productivity to train the new
people. Such a sacrifice will be devastating to a late project. The late project
will be even more behind schedule by the time the new engineers reach the
desirable efficiency. In short, it is not feasible to hire new engineers to
rescue a late project.

In his book, Brooks also includes several other interesting factors that may
impact the progress of a project, such as inter-employee communication, planning
and optimism. In our paper, we will incorporate these ideas to develop a
mathematic model that captures the dynamics of a project and employees working
for the project.

\section*{Base Case}
    \begin{atomize}
        \item Assumptions:
            \begin{itemize}
								\item Brooks' law can be modelled as the relationships between
								time, junior staff, senior staff, and progress.
								\item The relationships between these quantites can be modelled
								as a system of first order differential equations.
								\item Any of these quatities could effect any other quantity
								besides time.  
								\end{itemize}
        \item Quantities:
            \begin{itemize}
                \item [$t$]: time spent working
                \item [$x$]: junior staff
                \item [$y$]: senior staff
                \item [$p$]: progress
            \end{itemize}
        \item Model:
            \begin{atomize}
                \item $x\prime = f\left(x, y, p, t\right)$
                \item $y\prime = g\left(x, y, p, t\right)$
                \item $p\prime = h\left(x, y, p, t\right)$
            \end{atomize}
    \end{atomize}

\section*{Case I: extends Base}
    \begin{atomize}
        \item Assumptions:
            \begin{itemize}
                \item Only Junior staff is added.
                \item Only Senior staff contributes progress.
								\item Junior staff is added as a function of time, at a constant
								rate.
								\item Junior staff becomes senior staff as a function of time,
								at a constant rate.
								\item Senior staff contribute progress as a function of time, at
								a constant rate.
            \end{itemize}
        \item Quantities:
            \begin{itemize}
                \item [$r$]: rate at which junior staff is added
                \item [$s$]: rate at which junior staff becomes senior staff
                \item [$k$]: rate at which senior staff contributes progress
            \end{itemize}
        \item Model:
            \begin{atomize}
                \item $x\prime = r - s$
                \item $y\prime = sx$
                \item $p\prime = ky$
            \end{atomize}
        \item Solution:
            \begin{atomize}
                \item $x = \left(r - s\right)t + x_{0}$
                \item $y = s\left(r - s\right)t + sx_{0}$
								\item $p = \frac{1}{6}ks(r-s)t^{3} + \frac{1}{2}ksx_{0}t^{2} +
								ky_{0}t + p_{0}$
            \end{atomize}
        \item Interpret:
            \begin{itemize}
                \item Bullshit!
            \end{itemize}
    \end{atomize}

\section*{Case II: extends Case I}
    \begin{atomize}
        \item Assumptions:
            \begin{itemize}
								\item Junior staff inhibits the senior staff ability to
								contribute progress as a function of time, at a constant rate.
            \end{itemize}
        \item Quantities:
            \begin{itemize}
								\item [$l$]: rate at which junior staff inhibits the senior
								staff ability to contribute progress
            \end{itemize}
        \item Model:
            \begin{atomize}
                \item $x\prime = r - s$
                \item $y\prime = sx$
                \item $p\prime = ky - lx$
            \end{atomize}
        \item Solution:
            \begin{atomize}
                \item $x = \left(r - s\right)t + x_{0}$
                \item $y = s\left(r - s\right)t + sx_{0}$
								\item $p = \frac{1}{6}ks(r-s)t^{3} + \frac{1}{2}ksx_{0}t^{2} +
								ky_{0}t - \frac{1}{2}l(r-s)t^{2} - lx_{0}t  + p_{0} $
            \end{atomize}
        \item Interpret:
            \begin{itemize}
                \item More bullshit!
            \end{itemize}
    \end{atomize}

\end{document}

