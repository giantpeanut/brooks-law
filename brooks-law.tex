\documentclass{article}

% margins
\usepackage[top=1in, bottom=1.5in, left=1.5in, right=1.5in]{geometry}

% smart quotes
\usepackage [english]{babel}
\usepackage [autostyle, english = american]{csquotes}
\MakeOuterQuote{"}

% line spacing
\usepackage{setspace}
\singlespacing
\usepackage[fleqn]{amsmath}

% images
\usepackage{graphicx}
\DeclareGraphicsExtensions{.png}
\graphicspath{ {./images/} }

% for H option for figure environment
% H forces figure to be placed where written
\usepackage{float}

% capital Tau
\newcommand{\Tau}{\mathrm{T}}

% tiny bullet points
\renewcommand{\labelitemi}{$\vcenter{\hbox{\tiny$\bullet$}}$}

% new list environment for lists with no labels
% documentation on how this works at:
%     run `exdoc source2e`
%     look at section about lists on page 218 
\newenvironment{atomize}
    {\begin{list} {} {
            \setlength\itemindent{0pt}
            \setlength\leftmargin{10pt}
            \setlength\labelwidth{0pt}
    }}
    {\end{list}}

% info for \maketitle
\author{Dylan Holmes, Kuangyou Yao, Jonathan K}
\title{Brooks' law}
\date{February 17$^{th}$, 2014}

% todo checklist
% - - — - - - - - - - - - - — - - - - - - - - - - — - - - - - - - - - - 
% [ ] write more
% - - — - - - - - - - - - - — - - - - - - - - - - — - - - - - - - - - - 

\begin{document}

\maketitle

\section*{Abstract}
According to Brooks' law, "adding manpower to a late software project makes it
later". Brooks' law originated from Fred Brooks' 1975 book, "The Mythical
Man-Month", and has since become commonly accepted through out the software
development community. 

Suppose in a software firm, a project manager is facing a deadline in one month.
Given that the project is only 60\% done, the project manager faces the problem
of deciding if he should hire additional engineers to pace up the process.

Brook’s law argues that by adding new engineers, experienced and efficient
engineers will have to give up part of their productivity to train the new
people. Such a sacrifice will be devastating to a late project. The late project
will be even more behind schedule by the time the new engineers reach the
desirable efficiency. In short, it is not feasible to hire new engineers to
rescue a late project.

In his book, Brooks also includes several other interesting factors that may
impact the progress of a project, such as inter-employee communication, planning
and optimism. In our paper, we will incorporate some of these ideas to develop a
mathematic model that captures the dynamics of a project and employees working
for the project. By non-dimentionalizing each case we are able to obtain a 
simple version of the system in which brooks-law is observed to be
true for t subject to an associated inequality.

\section*{Base Case}
  \begin{atomize}
    \item Assumptions:
      \begin{itemize}
        \item Brooks' law can be modeled as the relationships between
        time, junior staff, senior staff, and progress.
        \item The relationships between these quantities can be modeled
        as a system of first order differential equations.
        \item Any of these quantities could effect any other quantity besides
        time.  
        \end{itemize}

    \item Quantities:
      \begin{itemize}
        \item [$t$]: time spent working in days
        \item [$x$]: junior staff in number of people
        \item [$y$]: senior staff in number of people
        \item [$p$]: progress in of lines of code (added, deleted, or revised)
        \item [$t_{d}$]: target deadline
        \item [$*_{0}$]: initial value of the quantity "*"
        \item [$*_{s}$]: value of the quantity "*" without hiring
        \item [$*_{j}$]: value of the quantity "*" without hiring
      \end{itemize}
  
    \item Model:
      \begin{align*}
        &x^{\prime} = f\left(x, y, p, t\right) \\[6pt]
        &y^{\prime} = g\left(x, y, p, t\right) \\[6pt]
        &p^{\prime} = h\left(x, y, p, t\right)
      \end{align*}
  \end{atomize}

\section*{Case I: extends Base}
  \begin{atomize}
    \item Assumptions:
      \begin{itemize}
        \item Only Junior staff is added.
        \item Only Senior staff contributes progress.
        \item Junior staff is added as a function of time, at a constant rate.
        \item Junior staff becomes senior staff as a function of time, at a
        constant rate.
        \item Senior staff contribute progress as a function of time, at a
        constant rate.
        \item Junior staff cannot become senior staff faster than they are
        added.
        \item The project is already under way and a deadline is approaching.
        \item Until now, no junior staff has been added.
      \end{itemize}

    \item Quantities:
      \begin{itemize} 
        \item [$r$]: rate at which junior staff is added in people per day 
        \item [$s$]: rate at which junior staff becomes senior staff in people per day
        \item [$k$]: rate at which senior staff contributes progress in lines of
        code per person per day
      \end{itemize}

    \item Conditions:
      \begin{align*}
        &x_{0} = 0 \\[6pt]
        &r \geq s
      \end{align*}

    \item Model:
      \begin{align*}
        &x^{\prime} = r - s \\[6pt]
        &y^{\prime} = sx \\[6pt]
        &p^{\prime} = ky
      \end{align*}

    \item Interpret:
      \begin{atomize}
        \item Let
          \begin{align*}
            &t = \tau \Tau \\[6pt]
            &x = \xi\Xi \\[6pt]
            &y = \phi\Phi \\[6pt]
            &p = \psi\Psi
          \end{align*}

        \item Nondimensionalized, the model becomes:
          \begin{align*}
            \Xi^{\prime} &= \frac{\tau}{\xi}(r - s) \\[6pt]
            \Phi^{\prime} &= \frac{s\tau\xi}{\phi}\Xi \\[6pt]
            \Psi^{\prime} &= \frac{k\phi\tau}{\psi}\Phi
          \end{align*}

        \item Let
          \begin{align*}
            \tau = \xi = \phi = \psi = \frac{1}{k}
          \end{align*}
          
        \item After substituting:
          \begin{align*}
            \Xi^{\prime} &= r - s \\[6pt]
            \Phi^{\prime} &= \frac{s}{k}\Xi \\[6pt]
            \Psi^{\prime} &= \Phi
          \end{align*}

        \item Solving in terms of $\Psi$:
          \begin{align*}
            &\Psi^{\prime} = \Phi \\[6pt]
            &\Psi^{\prime\prime} = \Phi^{\prime} = \frac{s}{k}\Xi \\[6pt]
            &\Psi^{\prime\prime\prime} = \frac{s}{k}\Xi^{\prime} = 
            \frac{s}{k}(r - s) \\[6pt]
          \end{align*}

        \item Let
          \begin{align*}
            &\alpha = \frac{s}{k}(r-s), \alpha \geq 0
          \end{align*}

        \item Now, in terms of $\Psi^{\prime\prime\prime}$:
          \begin{align*}
            &\Psi^{\prime\prime\prime} = \alpha \\[6pt]
            &\Psi^{\prime\prime}(0) = \Xi_{0} = \frac{x_{0}}{\xi} = 0 \\[6pt]
            &\Psi^{\prime}(0) = \Phi_{0} \\[6pt]
            &\Psi(0) = \Psi_{0}
          \end{align*}

        \item $\Psi$ is nondimensionalized progress. Since both strategies start
        out with the same initial conditions, $\Psi^{\prime\prime\prime}$ 
        is only factor that differentiates the two strategies.  With hiring
        $\Psi^{\prime\prime\prime} > 0$, meaning that the rate of progress will
        increase and without hiring $\Psi^{\prime\prime\prime} = 0$, meaning
        that progress will continue at the same constant rate. Thus, hiring
        will always be more effective than not hiring. 
        \item 
          \begin{figure}[H]
            \centering
            \includegraphics[width=\textwidth]{caseI}
          \end{figure}
      \end{atomize}

    \item Reality:
      \begin{atomize}
        \item This model does not model Brooks' law under any conditions.
        Hiring new staff with always be more a more effective strategy, no
        matter how soon the deadline is. Brooks' law predicts that hiring
        new staff is a less effective strategy with a sufficiently close
        deadline.  At the very minimum, the model must have some conditions
        under which hiring new staff would slow progress and thus, cause a
        late project to be even later. One important factor not taken into
        consideration was the time taken by senior staff to train junior
        staff, which we will incorporate into the next iteration of this
        model.
      \end{atomize}
  \end{atomize}

\section*{Case II: extends Case I}
  \begin{atomize}
    \item Assumptions:
      \begin{itemize}
        \item Junior staff inhibits the senior staff's ability to contribute
        progress as a function of time, at a constant rate.
      \end{itemize}

    \item Quantities:
      \begin{itemize}
        \item [$\ell$]: rate at which junior staff inhibits the senior staff
        ability to contribute progress
      \end{itemize}

    \item Conditions:
      \begin{atomize}
        \item $\ell \geq 0$
      \end{atomize}

    \item Model:
      \begin{align*}
        &x^{\prime} = r - s \\[6pt]
        &y^{\prime} = sx \\[6pt]
        &p^{\prime} = ky - lx
      \end{align*}

    \item Interpret:
      \begin{atomize}
        \item Let
          \begin{align*}
            &t = \tau \Tau \\[6pt]
            &x = \xi\Xi \\[6pt]
            &y = \phi\Phi \\[6pt]
            &p = \psi\Psi
          \end{align*}

        \item Nondimensionalized, the model becomes:
          \begin{align*}
            \Xi^{\prime} &= \frac{\tau}{\xi}(r - s) \\[6pt]
            \Phi^{\prime} &= \frac{s\tau\xi}{\phi}\Xi \\[6pt]
            \Psi^{\prime} &= \frac{k\phi\tau}{\psi}\Phi - 
              \frac{\ell\xi\tau}{\psi}\Xi
          \end{align*}

        \item Let
          \begin{align*}
            \tau = \xi = \phi = \psi = \frac{1}{k}
          \end{align*}
          
        \item After substituting:
          \begin{align*}
            \Xi^{\prime} &= r - s \\[6pt]
            \Phi^{\prime} &= \frac{s}{k}\Xi \\[6pt]
            \Psi^{\prime} &= \Phi - \frac{\ell}{k}\Xi
          \end{align*}

        \item Solving in terms of $\Psi$:
          \begin{align*}
            &\Psi^{\prime} = \Phi - \frac{\ell}{k}\Xi \\[6pt]
            &\Psi^{\prime\prime} = \Phi^{\prime} - \frac{\ell}{k}\Xi^{\prime} 
            = \frac{s}{k}\Xi - \frac{\ell}{k}(r - s) \\[6pt]
            &\Psi^{\prime\prime\prime} = \frac{s}{k}\Xi^{\prime} = 
            \frac{s}{k}(r - s) \\[6pt]
          \end{align*}

        \item Let
          \begin{align*}
            &\alpha = \frac{1}{k}(r-s), \alpha \geq 0 \\[6pt]
            &\beta = s\alpha, \beta \geq 0 \\[6pt]
            &\gamma = -\ell\alpha, \gamma \leq 0
          \end{align*}

        \item Now, in terms of $\Psi^{\prime\prime\prime}$:
          \begin{align*}
            &\Psi^{\prime\prime\prime} = \beta \\[6pt]
            &\Psi^{\prime\prime}(0) = \frac{s}{k}\Xi_{0} + \gamma = 
              \frac{x_{0}}{\xi} + \gamma = \gamma \\[6pt]
            &\Psi^{\prime}(0) = \Phi_{0} \\[6pt]
            &\Psi(0) = \Psi_{0}
          \end{align*}

        \item $\Psi$ is nondimensionalized progress. This time we have
        two parameters to consider. Since both of these are a combination of
        our physical parameters, we can't confidently attribute any of the 
        effects that we see to a specific parameters, but it does however 
        allow us to clearly see that there are conditions under which brooks 
        law is true. When ever $\gamma < 0$ we see $\Psi$ dip slightly below 
        the non-hiring case (when $\gamma = \beta = 0$).
        
        \item 
          \begin{figure}[H]
            \centering
            \includegraphics[width=\textwidth]{caseII}
          \end{figure}

        \item To find the conditions under which this case obeys Brooks's law,
        we want to find some relationship between the physical
        parameters that will cause the late project to be later, which
        happens mathematically when $p_{j}(t_{d}) < p_{s}(t_{d})$.

        \item First, let's find $p(t)$
          \begin{align*}
            x^{\prime} &= r - s \\[6pt]
            \int^{t}_{t_{0}} x^{\prime} dt &= (r - s)\int^{t}_{t_{0}}dt \\[6pt]
            x(t) &= (r - s)t \\[6pt]
            \\[3pt]
            y^{\prime} &= sx \\[6pt]
            y^{\prime} &= s(r - s)t \\[6pt]
            \int^{t}_{t_{0}} y^{\prime} dt &= s(r - s)\int^{t}_{t_{0}} t dt
            \\[6pt]
            y(t) &= \frac{1}{2}s(r - s)t^{2} + y_{0}\\[6pt]
            \\[3pt]
            p^{\prime} &= ky - {\ell}x \\[6pt]
            p^{\prime} &= k\left(\frac{1}{2}s(r - s)t^{2} + y_{0}\right) 
            - \ell(r - s)t \\[6pt]
            \int^{t}_{t_{0}} p^{\prime} dt &= \int^{t}_{t_{0}}
            \left(\frac{ks}{2}(r - s)t^{2} - \ell(r - s)t + ky_{0}\right)
            dt \\[6pt]
            p(t) &= \frac{1}{6}ks(r - s)t^{3} - \frac{1}{2}\ell(r - s)t^{2} 
            + ky_{0}t  + p_{0}
          \end{align*}

        \item Now, $p_{s}(t_{d})$ and $p_{j}(t_{d})$
          \begin{align*}
            p_{s}(t_{d}) &= ky_{0}t_{d}  + p_{0} \\[6pt]
            p_{j}(t_{d}) &= \frac{1}{6}ks(r - s)t_{d}^{3} 
            - \frac{1}{2}\ell(r - s)t_{d}^{2} + ky_{0}t_{d}  + p_{0}
          \end{align*}

        \item Finally set $p_{j}(t_{d}) < p_{s}(t_{d})$, where 
        $t_{d} > t_{0} = 0$
          \begin{align*}
            p_{j}(t_{d}) &< p_{s}(t_{d}) \\[6pt]
            \frac{1}{6}ks(r - s)t_{d}^{3} - \frac{1}{2}\ell(r - s)t_{d}^{2}
            + ky_{0}t_{d}  + p_{0} &< ky_{0}t_{d} + p_{0} \\[6pt]
            t_{d}^{2}\left(\frac{1}{6}ks(r - s)t_{d}
            - \frac{1}{2}\ell(r - s)\right) &< 0 \\[6pt]
            \frac{1}{6}ks(r - s)t_{d}
            &< \frac{1}{2}\ell(r - s) \\[6pt]
            t_{d} &< \frac{3\ell}{ks} \\[6pt]
          \end{align*}

      \end{atomize}

    \item Reality:
      \begin{atomize}
        \item This model has the potential to model brooks law, because there
          are conditions under which Brooks' law holds true, specifically it
          holds true when $t_{d} < \frac{3\ell}{ks}$.

          \item This infers a few interesting relationships about the 
          break-even deadline (when $t_{d} = \frac{3\ell}{ks}$), that it is:
            \begin{itemize}
              \item directly proportional rate at which junior staff inhibits
              progress
              \item inversely proportional to the rate at which senior
              staff contribute progress
              \item inversely proportional to the rate at which junior
              staff becomes senior staff
            \end{itemize}
          These relationships can be explained by the fact that all of these 
          rates are constant and that:
            \begin{itemize}
              \item junior staff inhibiting progress should set the break-even
              deadline farther into the future, and that if they don't inhibit
              progress($\ell = 0$), the break-even point is essentially the
              present ($t_{0}$).
              \item increasing senior productive should set the break-deadline
              closer to the present ($t_{0}$).
              \item increasing senior staff should also set the break-deadline
              closer to the present ($t_{0}$).
            \end{itemize}

          Although this model the effect described by Brooks' law, it doesn't
          model a real software development project. One of the more obvious
          flaw of the model is that it is unreasonable to assume that the rate
          progress will continue increase without bound.
                    
          We still have not taken into account the fact that a project will
          generally become more complex as it increases in size. For example,
          working on a project with one hundred lines of code and one with over
          a million would be a very different task. You could easily read
          through the smaller project in most likely less than an hour and be
          able to added to it quickly with relatively little chance of
          introducing a bug or getting confused. The larger project however,
          would most likely take much more time to read, and work with, not to
          mention the lengthier time it would take to debug. Of course there 
          are other factors affecting a projects complexity, but we will start
          simply, by considering the effects of project size in our next case.
      \end{atomize}

  \end{atomize}

  \pagebreak
	\section*{Case III: extends Case II}
		\begin{atomize}
			\item Assumptions:
				\begin{itemize}
					\item The rate of progress will decrease, as a function of total
					progress, at a constant rate.
				\end{itemize}

			\item Quantities
				\begin{itemize}
					\item [$m$]: rate at which total progress decreases the rate of
					progress
				\end{itemize}

			\item Model:
				\begin{atomize}
					\item $x^{\prime} = r - s $
					\item $y^{\prime} = sx $
					\item $p^{\prime} = ky - lx - mp $
				\end{atomize}

			\item Nondimensionalizing:
				\begin{atomize}
					\item Set:
							\begin{align*}
								t &= \tau \Tau \\[6pt]
								x &= \xi \Xi  \\[6pt]
								y &= \phi \Phi \\[6pt]
								p &= \psi \Psi \\[6pt]
							\end{align*}

					\item Transformed: 
							\begin{align*}
								\Xi^{\prime} &= \frac{\Tau}{\xi} (r-s) \\[6pt]
								\Phi^{\prime} &= \frac{\Tau}{\phi} s \xi \Xi \\[6pt]
								\Psi^{\prime} &= \frac{\Tau}{\psi} (k \phi \Phi - l \xi \Xi 
                - m\psi \Psi)
							\end{align*}

					\item Finalize: setting $\xi = \phi = \psi = \Tau = \frac{1}{k} $
							\begin{align*}
								\Xi^{\prime} &= r - s \\[6pt]
								\Phi^{\prime} &= \frac{s}{k} \Xi \\[6pt]
								\Psi^{\prime} &= \Phi - \frac{m}{k} \Psi - \frac{l}{k} \Xi  	
							\end{align*}

					\item Solving: 
							\begin{atomize}
								\item $\Xi = (r-s)\tau + \Xi_{0}$
								\item $\Phi = \frac{s(r-s)}{2k}\tau^{2} + \frac{s}{k}\Xi_{0}\tau + \Phi_{0}$
								\item I assume that $k = 30; l = 10; m = 14; \Xi_{0} = 0; \Phi_{0} = 3$ 
								\item Case 1 is when the firm is not hiring new staff $(r = s =
								0)$
								\item Case 2 is when the firm hires new staff and half of the
								junior staff turn to senior staff $(r=10; s=5)$
                \item We got the following graph:
							\end{atomize}
 		       \item 
    		      \begin{figure}[H]
        	    \centering
          	  \includegraphics[width=\textwidth]{caseIII}
         	 \end{figure}

          \item Interpretation:
            \begin{itemize}
							\item This is exactly what brooks law is showing, that the new
							staff will actually slow down the overall progress. We can see
							that the before the intersection, the progress is lower. That
							signifies that junior staff are still adjusting to the new
							environment, and senior staff are sacrifising their output to
							coach the junior staff. 

							\item This also captures the fact that when the firm is not hiring
							new staff, the bigger size the project is, the slower the project
							progresses. This is an improvement from the previous model.

							\item Unfortunately, in the case when the firm is hiring new
							staff, this model does not capture the fact that the bigger the
							project, the slower the project should progress. In the graph, the
							project just spike up faster and faster. We believe that is
							because that the number of staff is not bounded in the assumption.
							So even with a large m (damping factor of the size of the
							project), we still only get the green line when $m=400$, which is
							going to spike up like the red line given enough amount of time. 

            \end{itemize}


				\end{atomize}
			
		\end{atomize}
\pagebreak
  \section*{Conclusion}
   In conclusion, we found 3 possiblilities for which brooks-law would be true
   for each case's inequalities.
   The aquisition of empirical data is outside the scope of this report but the
   following results will partially describe the real scenario.
   \begin{atomize}
    \item Case I: Any policy for hiring new employees will results in a earlier project completion. 
    \item Case II: will pertain to brooks-law if the hiring policies follow: 
    $t_{d} < \frac{3\ell}{ks}$ [6pt]
    \item Case III: Observing the non-dimentionalized $\Psi$, we found that brooks-law is somewhat true in case III since there exists conditions where the rate of progress actually slow down then turn back up. However it just increases faster and faster after a certain point, which is a point for improvement in future studies. 
  \end{atomize}
\end{document}