\documentclass{article}

% margins
\usepackage[top=1in, bottom=1.5in, left=1.5in, right=1.5in]{geometry}

% smart quotes
\usepackage [english]{babel}
\usepackage [autostyle, english = american]{csquotes}
\MakeOuterQuote{"}

% line spacing
\usepackage{setspace}
\singlespacing

\usepackage[fleqn]{amsmath}

% images
\usepackage{graphicx}
\DeclareGraphicsExtensions{.png}
\graphicspath{ {./images/} }

% for H option for figure environment
% H forces figure to be placed where written
\usepackage{float}

% capital Tau
\newcommand{\Tau}{\mathrm{T}}

% tiny bullet points
\renewcommand{\labelitemi}{$\vcenter{\hbox{\tiny$\bullet$}}$}

% new list environment for lists with no labels
% documentation on how this works at:
%     run `texdoc source2e`
%     look at section about lists on page 218 
\newenvironment{atomize}
    {\begin{list} {} {
            \setlength\itemindent{0pt}
            \setlength\leftmargin{10pt}
            \setlength\labelwidth{0pt}
    }}
    {\end{list}}

% info for \maketitle
\author{Dylan Holmes, Kuangyou Yao, Jonathan K}
\title{Brooks' law}
\date{February 17$^{th}$, 2014}

% todo checklist
% - - — - - - - - - - - - - — - - - - - - - - - - — - - - - - - - - - - 
% [ ] write more
% - - — - - - - - - - - - - — - - - - - - - - - - — - - - - - - - - - - 

\begin{document}

\maketitle

\section*{Abstract}
According to Brooks' Law, "adding manpower to a late software project makes it
later". Brooks' law originated from Fred Brooks' 1975 book, "The Mythical
Man-Month", and has since become commonly accepted through out the software
development community. 

Suppose in a software firm, a project manager is facing a deadline in one month.
Given that the project is only 60\% done, the project manager faces the problem
of deciding if he should hire additional engineers to pace up the process.

Brook’s law argues that by adding new engineers, experienced and efficient
engineers will have to give up part of their productivity to train the new
people. Such a sacrifice will be devastating to a late project. The late project
will be even more behind schedule by the time the new engineers reach the
desirable efficiency. In short, it is not feasible to hire new engineers to
rescue a late project.

In his book, Brooks also includes several other interesting factors that may
impact the progress of a project, such as inter-employee communication, planning
and optimism. In our paper, we will incorporate these ideas to develop a
mathematic model that captures the dynamics of a project and employees working
for the project.

\section*{Base Case}
  \begin{atomize}
    \item Assumptions:
      \begin{itemize}
				\item Brooks' law can be modelled as the relationships between
				time, junior staff, senior staff, and progress.
				\item The relationships between these quantites can be modelled
				as a system of first order differential equations.
				\item Any of these quantities could effect any other quantity besides
				time.  
        \end{itemize}

    \item Quantities:
      \begin{itemize}
        \item [$t$]: time spent working in days
        \item [$x$]: junior staff in number of people
        \item [$y$]: senior staff in number of people
        \item [$p$]: progress in of lines of code (added, deleted, or revised)
        \item [$t_{0}$]: initial time
        \item [$x_{0}$]: initial junior staff
        \item [$y_{0}$]: initial senior staff
        \item [$p_{0}$]: initial progress
        \item [$t_{d}$]: target deadline
				\item [$x_{s}$]: junior staff without hiring
				\item [$y_{s}$]: senior staff without hiring
        \item [$p_{s}$]: progress without hiring
        \item [$x_{j}$]: junior staff with hiring
				\item [$y_{j}$]: senior staff with hiring
        \item [$p_{j}$]: progress with hiring
      \end{itemize}
	
    \item Model:
      \begin{align*}
        &x\prime = f\left(x, y, p, t\right) \\[6pt]
        &y\prime = g\left(x, y, p, t\right) \\[6pt]
        &p\prime = h\left(x, y, p, t\right)
      \end{align*}
  \end{atomize}

\section*{Case I: extends Base}
  \begin{atomize}
    \item Assumptions:
      \begin{itemize}
        \item Only Junior staff is added.
				\item Only Senior staff contributes progress.
				\item Junior staff is added as a function of time, at a constant rate.
				\item Junior staff becomes senior staff as a function of time, at a
				constant rate.
				\item Senior staff contribute progress as a function of time, at a
				constant rate.
				\item Junior staff cannot become senior staff faster than they are
				added.
				\item The project is already under way and a deadline is approaching.
				\item Until now, no junior staff has been added.
      \end{itemize}

		\item Quantities:
			\begin{itemize} 
				\item [$r$]: rate at which junior staff is added in people per day 
				\item [$s$]: rate at which junior staff becomes senior staff in people per day
				\item [$k$]: rate at which senior staff contributes progress in lines of
				code per person per day
			\end{itemize}

		\item Conditions:
  		\begin{align*}
  	  	&x_{0} = 0 \\[6pt]
  	  	&r > s
  		\end{align*}

		\item Model:
      \begin{align*}
        &x\prime = r - s \\[6pt]
        &y\prime = sx \\[6pt]
        &p\prime = ky
      \end{align*}
			\pagebreak

		\item Interpret:
			\begin{atomize}
				\item Let
					\begin{align*}
						&t = \tau \Tau \\[6pt]
						&x = \xi\Xi \\[6pt]
						&y = \phi\Phi \\[6pt]
						&p = \psi\Psi
					\end{align*}
				\item The model becomes
					\begin{align*}
						\Xi^{\prime} &= \frac{\tau}{\xi}(r - s) \\[6pt]
						\Phi^{\prime} &= \frac{s\tau\xi}{\phi}\Xi \\[6pt]
						\Psi^{\prime} &= \frac{k\phi\tau}{\psi}\Phi
					\end{align*}
				\item Let
					\begin{align*}
						\tau = \xi = \phi = \psi = \frac{1}{k}
					\end{align*}
				\item The model becomes
					\begin{align*}
						\Xi^{\prime} &= r - s \\[6pt]
						\Phi^{\prime} &= \frac{s}{k}\Xi \\[6pt]
						\Psi^{\prime} &= \Phi
					\end{align*}
				\item Solving in terms of $\Psi$:
					\begin{align*}
						&\beta = \frac{s}{k}(r-s), \beta \geq 0 \\[6pt]
						&\Psi^{\prime\prime\prime} = \beta \\[6pt]
						&\Psi^{\prime\prime}(0) = \Xi_{0} = \frac{x_{0}}{\xi} = 0 \\[6pt]
						&\Psi^{\prime}(0) = \Phi_{0} \\[6pt]
						&\Psi(0) = \Psi_{0}
					\end{align*}
				\item $\Psi$ is nondimensionalized progress. Since both stategies start
				out with the same intial conditions, $\Psi^{\prime\prime\prime}$ is only
				factor that differentiates the two strategies.  With hiring
				$\Psi^{\prime\prime\prime} > 0$, meaning that the rate of progress will
				increase and without hiring $\Psi^{\prime\prime\prime} = 0$, meaning
				that progress will continue at the same constant rate. Thus, hiring will
				always be more effective than not hiring. 
				\item 
					\begin{figure}[H]
						\centering
						\includegraphics[width=\textwidth]{caseI}
					\end{figure}
			\end{atomize}

		\item Reality:
			\begin{atomize}
				\item This model does not model Brooks' Law under any conditions.
				Hiring new staff with always be more a more effective strategy, no
				matter how soon the deadline is. Brooks' Law predicts that hiring
				new staff is a less effective strategy with a sufficently close
				deadline.  At the very minimum, the model must have some conditions
				under which hiring new staff would slow progress and thus, cause a
				late project to be even later. One important factor not taken into
				consideration was the time taken by senior staff to train junior
				staff, which we will incorporate into the next iteration of this
				model.
			\end{atomize}
  \end{atomize}

\section*{Case II: extends Case I}
  \begin{atomize}
    \item Assumptions:
      \begin{itemize}
				\item Junior staff inhibits the senior staff ability to contribute
				progress as a function of time, at a constant rate.
      \end{itemize}

    \item Quantities:
      \begin{itemize}
				\item [$l$]: rate at which junior staff inhibits the senior staff
				ability to contribute progress
      \end{itemize}

    \item Model:
      \begin{atomize}
        \item $x\prime = r - s$
        \item $y\prime = sx$
        \item $p\prime = ky - lx$
      \end{atomize}

    \item Solution:
      \begin{atomize}
        \item $x = \left(r - s\right)t + x_{0}$
        \item $y = \frac{1}{2}s\left(r - s\right)t^{2} + sx_{0}t+y_{0}$
				\item $p = \frac{1}{6}ks(r-s)t^{3} + \frac{1}{2}ksx_{0}t^{2} + ky_{0}t -
				\frac{1}{2}l(r-s)t^{2} - lx_{0}t  + p_{0} $
      \end{atomize}

    \item Interpret:
      \begin{itemize}
				\item Using the analysis above, if we try to find the intersecting point
				of the curve $s=0$ and $s\neq0$, we find that
				t=$\frac{31r-31s-3ksx_{0}\pm\sqrt{481(krs-ks^2)x_{0} +
				(-31r+31s+3ksx_{0})^2}}{2(krs-ks^2)}$. 
        \item We see that t is .... Oh god do this later...
      \end{itemize}
  \end{atomize}

\end{document}

